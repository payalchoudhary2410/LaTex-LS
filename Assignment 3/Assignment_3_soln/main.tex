\documentclass{article}
\usepackage[utf8]{inputenc}
\usepackage{chemfig}
\usepackage{modiagram}
\usepackage[american]{circuitikz}



\title{Field-Specific \LaTeX{} tricks }
\author{Payal Choudhary }
\date{\today}

\begin{document}

\maketitle

\tableofcontents

\newpage

\section{Chemistry}
\subsection{Chemical Formulae}
\chemfig{
               Cl% 11
         -[:30]% 5
        =_[:90]% 4
         -[:30]% 3
       =_[:330]% 2
        -[:270]% 1
                  (
           =_[:210]% 6
            -[:150]% -> 5
                  )
        -[:330]N% 7
        =^[:30]% 8
         -[:90]% 9
       =^[:150]% 10
                  (
            -[:210]% -> 2
                  )
     -[:90,,,2]HN% 12
      -[:30,,2]% 13
                  (
             -[:90]% 15
                  )
        -[:330]% 14
         -[:30]% 16
        -[:330]% 17
         -[:30]N% 18
                  (
            -[:330]% 19
             -[:30]% 23
                  )
         -[:90]% 20
         -[:30]% 21
    -[:330,,,1]OH% 22
}
\vspace{0.5cm}



\begin{flushleft}
This is the molecule hydroxychloroquine, that recently shot to fame as a proposed cure for COVID-19. Please draw it. This
is a helpful Overleaf tutorial to help you get started.
\end{flushleft}


\subsection{Molecular Orbital Diagrams}
\begin{MOdiagram}[labels,names]
 \atom[N]{left}{
   2p = {0;up,up,up}
 }
 \atom[O]{right}{
   2p = {2;pair,up,up}
 }
 \molecule[NO]{
   2pMO  = {1.8,.4;pair,pair,pair,up}  }
\end{MOdiagram}

\vspace{0.5cm}
\begin{flushleft}
You’ve probably mugged this up for JEE, and definitely learnt more about this in CH 107.
Draw the above molecular orbital diagram for nitric oxide. Again, exact dimensions needn’t match.
\end{flushleft}



\newpage

\section{Electrical circuits}
\vspace{0.5cm}
\begin{circuitikz}[scale=2]
 \draw[color=black, thick]
(0,0) to [short,o-o] (5,0)
to [short,o-o] (6,0)
(0,1) node[]{\large{$V_o{sin{(\omega t)}}$}}
(4,0) node[ground]{} node[circ](5,0){}
(0,2) to [C, l=$C_1$, o-] (2,2)
to node[short]{}(2.8,2)
(2,2) to [R, l=$R_1$, -] (2,5)
 to [short,*-o](6,5) 
% Transistor Bipolar 
(3,0) to [R,l=$R_E$,v<=$V_E$,-] (3,1.2)
to [Tnpn,n=npn1] (3,2.8)
(npn1.E) node[right=6mm, above=5mm]{$V_{CE}$} % Labelling the NPN transistor
(4,0) to [C, l_=$C_4$, *-](4, 1.2)--(3,1.2)
(2,0) to [R, l=$R_2$,v<=$V_B$, -] (2,2)
(3,2.5) to node[short]{}(3,3)
(3,5) to [R, l_=$R_L$,i>=$i_C$, *-] (3,3)
to [short,*-o](5,3)

;
\end{circuitikz}
\begin{flushleft}


If you recall JEE Physics, this is a circuit diagram of an npn transistor used as an amplifier. Try your best to match this
circuit. We have used the American voltages convention. It is alright if you can’t get the dimensions to match. What is
important is that you know how to use circuitikz to draw circuits with the components used above, and mark voltages and
currents.
\end{flushleft}
\vspace{0.5cm}

\section{Typesetting exams}
I have been appointed as a TA for the course I loved last year .The prof, however, is busy with his research,
and wants me to typeset an \textbf{exam}.




\end{document}
